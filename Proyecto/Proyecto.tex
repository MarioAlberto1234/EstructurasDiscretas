\usepackage{titlesec}
\rhead{González Cisneros Mario Alberto}
\lhead{Codificación de Huffman}


\title{Reporte del Proyecto: Codicicación de Huffman}
\author{González Cisneros Mario Alberto\\ Número de cuenta: 323121583 \\ Facultad de Ciencias, UNAM}
\date{\today}

\begin{document}


\section{Introducción}
Este proyecto tiene como objetivo principalmente, conocer la codificación y decodificación de Huffman, así como explorar los fundamentos de lo que es la compresión de datos, sus aplicaciones en distintos tipos de archivos, y la implementación de algoritmos como Huffman.

\section{Desarrollo}
\subsection{Estrategias y enfoques}

\subsection{Entradas y salidas estándar}

\subsection{Actividades}
\begin{enumerate}[label=\textbf{\arabic*.}]
  \item Diferencia entre compresión con pérdida y sin pérdida.
  \item Investigación sobre compresión de texto, imágenes, audio y video.
  \item Explicación de códigos de longitud variable.
  \item Análisis binario de frases con y sin espacios usando el comando \texttt{xxd}.
  \item Compresión de frases usando los algoritmos que ya vimos.
  \item Comparación entre algoritmo estándar de Huffman y su variante.
  \item Importancia del árbol de Huffman para la decodificación.
  \item Compresión en videojuegos como lo es el GTA V para consolas de séptima generación.
\end{enumerate}

\section{Conclusiones}


\section{Referencias}
\begin{itemize}

\end{itemize}

\end{document}
