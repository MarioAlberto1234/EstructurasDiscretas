\usepackage{titlesec}
\rhead{González Cisneros Mario Alberto}
\lhead{Codificación de Huffman}


\title{Reporte del Proyecto: Codicicación de Huffman}
\author{González Cisneros Mario Alberto\\ Número de cuenta: 323121583 \\ Facultad de Ciencias, UNAM}
\date{\today}

\begin{document}


\section{Introducción}
Este proyecto tiene como objetivo principalmente, conocer la codificación y decodificación de Huffman, así como explorar los fundamentos de lo que es la compresión de datos, sus aplicaciones en distintos tipos de archivos, y la implementación de algoritmos como Huffman.

\section{Desarrollo}
El desarrollo del proyecto se llevó a cabo con estrategias de un avance progresivo, es decir, realizar paso a paso o día por día una cosa nueva e ir revisando y corrigiendo errores o mejorando de lo mismo.
\subsection{Entradas y salidas estándar}

\subsection{Actividades de documentación}
\textbf{\textit{1-¿Cuál es la diferencia entre la compresión con perdida y compresión sin perdida?}}
Unas de las principales diferencias entre la compresión con pérdida y sin pérdida es que con pérdida esta elimina permanentemente los datos de un archivo y en el caso de sin pérdida, este restaura y reconstruye los datos comprimdos. También al momento de saber cuándo utilizarlos ya que la compresnión con pérdida se da cuando la pérdida de informacion de archivos es aceptable y por la otra parte es cuando la pérdida de la información de archivos es inaceptable. Por último, otra gran diferencia es es sus aplicaciones ya que una trabaja o puede trabajar con imágenes, video y audio, y la otra con textos, imágenes y audio.
\textbf{\textit{2-¿Cómo se lleva a cabo la compresión de texto, imágenes, video y audio?}}
La compresión de texto, imágenes, videos y audio se realiza mediante algoritmos que reducen la cantidad de datos necesarios para representar la información, manteniendo la calidad lo más posible. Por ejemplo, en el caso de los\textbf{textos}, puede ser con técnicas comúnes como el de este proyecto que es Huffman, LZW, entre otras, para \textbf{imágenes} están las compresiones con pérdida y sin pérdida, para los \textbf{videos} son con algoritmos populares como H.264, H.265, VP9 entre otros, y en el caso de \textbf{audios} igualmente con comprensión con pérdida o sin pérdida.


\section{Conclusiones}

\section{Referencias}
\begin{itemize}
https://www.adobe.com/uk/creativecloud/photography/discover/lossy-vs-lossless.html
https://www.profesionalreview.com/2023/08/12/compresion-descompresion/
\end{itemize}

\end{document}
