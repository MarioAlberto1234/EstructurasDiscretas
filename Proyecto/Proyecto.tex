\documentclass{article} 
\usepackage[utf8]{inputenc}  
\usepackage{graphicx}
\graphicspath{{/home/magikzgnz/Imágenes/}}
\usepackage{fancyhdr}        
\usepackage{url}            
\usepackage{float}          

\pagestyle{fancy}
\rhead{González Cisneros Mario Alberto}
\lhead{Codificación de Huffman}

\title{Reporte del Proyecto: Codificación de Huffman}
\author{González Cisneros Mario Alberto\\ Número de cuenta: 323121583 \\ Facultad de Ciencias, UNAM}
\date{\today}

\begin{document}


\begin{center}
    \includegraphics[width=\textwidth,height=0.9\textheight]{unam.png} 
\end{center}

\maketitle

\section{Introducción}
Este proyecto tiene como objetivo principalmente, conocer la codificación y decodificación de Huffman, así como explorar los fundamentos de lo que es la compresión de datos, sus aplicaciones en distintos tipos de archivos, y la implementación de algoritmos como Huffman.

\section{Desarrollo}
El desarrollo del proyecto se llevó a cabo con estrategias de un avance progresivo, es decir, realizar paso a paso o día por día una cosa nueva e ir revisando y corrigiendo errores o mejorando de lo mismo.

\subsection{Entradas y salidas estándar}

\subsection{Actividades de documentación}
\textbf{\textit{1-¿Cuál es la diferencia entre la compresión con pérdida y compresión sin pérdida?}}  
Unas de las principales diferencias entre la compresión con pérdida y sin pérdida es que con pérdida esta elimina permanentemente los datos de un archivo y en el caso de sin pérdida, este restaura y reconstruye los datos comprimidos. También al momento de saber cuándo utilizarlos ya que la compresión con pérdida se da cuando la pérdida de información de archivos es aceptable y por la otra parte es cuando la pérdida de la información de archivos es inaceptable. Por último, otra gran diferencia es en sus aplicaciones ya que una trabaja o puede trabajar con imágenes, video y audio, y la otra con textos, imágenes y audio.

\textbf{\textit{2-¿Cómo se lleva a cabo la compresión de texto, imágenes, video y audio?}}  
La compresión de texto, imágenes, videos y audio se realiza mediante algoritmos que reducen la cantidad de datos necesarios para representar la información, manteniendo la calidad lo más posible. Por ejemplo, en el caso de los \textbf{textos}, puede ser con técnicas comunes como el de este proyecto que es Huffman, LZW, entre otras. Para \textbf{imágenes} están las compresiones con pérdida y sin pérdida, para los \textbf{videos} son con algoritmos populares como H.264, H.265, VP9 entre otros, y en el caso de \textbf{audios} igualmente con compresión con pérdida o sin pérdida.

\textbf{\textit{3-¿A qué se refiere el enfoque para la compresión de datos, códigos de longitud variable?}}  
Se refiere a asignar a cada símbolo una cantidad distinta de bits según su frecuencia de aparición, es decir los símbolos más comunes reciben códigos más cortos y los menos frecuentes los más largos. Esto permite reducir el tamaño total de la representación sin pérdida de información.

\textbf{\textit{4- Archivos de texto donde está una de mis frases favoritas, una con espacios y la otra sin espacio para saber su contenido en binario mediante el comando xxd y sus pesos en bytes}}  
\begin{figure}[H]
    \centering
    \includegraphics[width=0.5\textwidth]{archivos.png}
    \caption{Contenidos binarios de mi frase favorita con espacios y sin espacios así como sus pesos en bytes}
\end{figure}

\textbf{\textit{4-¿Cuál es la diferencia respecto a contenido binario de los dos archivos?}}  
La única diferencia respecto al contenido binario de los respectivos archivos es la presencia de bytes de espacio, por lo que hace que el archivo sea más grande y que su representación binaria tenga más bits.

\textbf{\textit{5- Comprime la frase del anterior inciso y otra frase más (puede ser una frase de una de tus canciones favoritas), usando los algoritmos vistos. Para el primer algoritmo adjunta su representación binaria de inicio y final después de utilizarlo.}}  
\begin{figure}[H]
    \centering
    \includegraphics[width=0.5\textwidth]{arbol1.png}
    \caption{Compresión primera frase}
\end{figure}

\begin{figure}[H]
    \centering
    \includegraphics[width=0.5\textwidth]{arbol2.png}
    \caption{Compresión segunda frase}
\end{figure}

\textbf{\textit{6- El algoritmo que se muestra en este proyecto es una variante del algoritmo estándar de Huffman • Investiga cómo es este el algoritmo estándar • Menciona las ventajas y desventajas que tiene nuestra variante del algoritmo de Huffman con el estándar • Elige una de tus frases para aplicar el algoritmo estándar (solo codificación).}}  
Este algoritmo estándar construye un árbol binario en base a las frecuencias de cada carácter que contenga nuestra cadena o texto, siendo la frecuencia más alta o importante el carácter que más veces se repite y las que menos se repiten tienen menos importancia. Alguna de las \textbf{ventajas} de esta variante es que es quizá menos compleja y que su implementación sea más sencilla, las \textbf{desventajas} podrían ser que no siempre logra la compresión mínima.  
\underline{La codificación se puede observar en las imágenes del inciso 5.}

\textbf{\textit{7- ¿Por qué es necesario tener un respectivo árbol de Huffman para decodificar una cadena de texto? ¿Qué pasa si no lo tengo?}}  
El árbol de Huffman es indispensable para decodificar porque contiene la asignación exacta de códigos binarios a cada símbolo, así que sin él, la cadena comprimida se vuelve imposible de reconstruir correctamente ya que no sabríamos dónde termina un símbolo y empieza otro.

\textbf{\textit{8- ¿De qué manera las estrategias de optimización empleadas en el videojuego Grand Theft Auto V para consolas de séptima generación (PS3/Xbox 360) reflejan los principios fundamentales de la compresión de datos en comparación con las de octava generación?}}  
En PS3 y Xbox 360 el Grand Theft Auto V usó optimizaciones que reflejan algunos de los principios de compresión de datos al reducir resolución, texturas y distancia de dibujado para que el juego cupiera en la limitada memoria y ancho de banda. En el caso de las consolas de nueva generación, estas aplicaron estrategias menos restrictivas con mayor fidelidad gráfica y densidad de datos, porque el hardware permitía manejar más información sin tanta compresión perceptual.

\section{Conclusiones}

\section{Referencias}
\begin{itemize}
    \item \url{https://www.adobe.com/uk/creativecloud/photography/discover/lossy-vs-lossless.html}
    \item \url{https://www.profesionalreview.com/2023/08/12/compresion-descompresion/}
\end{itemize}

\end{document}
